% !TeX TXS-program:compile = txs:///pdflatex

\documentclass[11pt,a4paper]{ltxdoc}
\usepackage{sim-os-menus}
\usepackage{graphicx}
\usepackage{fancyvrb}
\usepackage{fancyhdr}
\usepackage{hyperref}
\fancyhf{}
\renewcommand{\headrulewidth}{0pt}
%\rhead{\sffamily\small\affloetalab[Legende]}
\lfoot{\sffamily\small [sim-os-menus]}
\rfoot{\sffamily\small - \thepage{} -}
%\rfoot{\hyperlink{matoc}{\small\faArrowAltCircleUp[regular]}}
\usepackage{hologo}
\providecommand\tikzlogo{Ti\textit{k}Z}
\providecommand\TeXLive{\TeX{}Live\xspace}
\providecommand\PSTricks{\textsf{PSTricks}\xspace}
\let\pstricks\PSTricks
\let\TikZ\tikzlogo

\urlstyle{same}
\hypersetup{pdfborder=0 0 0}
\usepackage[margin=2cm]{geometry}
\setlength{\parindent}{0pt}
\def\TPversion{0.1.3}
\def\TPdate{28/09/2024}
\usepackage{soul}
\sethlcolor{lightgray!25}
\NewDocumentCommand\MontreCode{ m }{%
	\hl{\vphantom{\texttt{pf}}\texttt{#1}}%
}
\usepackage[french]{babel}

\begin{document}

\pagestyle{fancy}

\thispagestyle{empty}

\begin{center}
	\begin{minipage}{0.88\linewidth}
	\begin{tcolorbox}[colframe=yellow,colback=yellow!15]
		\begin{center}
			\begin{tabular}{c}
				{\Huge \texttt{sim-os-menus [fr]}}\\
				\\
				{\LARGE Présenter un 'terminal', une 'visionneuse'} \\
				{\LARGE ou un 'menu contextuel' comme sur un ordi.} \\
				\\
				{\small \texttt{Version \TPversion{} -- \TPdate}}
		\end{tabular}
		\end{center}
	\end{tcolorbox}
\end{minipage}
\end{center}

\begin{center}
	\begin{tabular}{c}
	\texttt{Cédric Pierquet}\\
	{\ttfamily c pierquet -- at -- outlook . fr}\\
	\texttt{\url{https://github.com/cpierquet/sim-os-menus}} \\
\end{tabular}
\end{center}

\hrule

\vfill

\begin{tcolorbox}[colframe=lightgray,colback=lightgray!5]
\MenuContextuel[Police=\sffamily,ListeSeps={1,3/1/4/1/},ListeDecals={1,2,1,0}]{%
	Menu1(>),Menu2,Menu3(>)(*),Menu4 §
	SousMenuA(>),SousMenuB(>)(*) §
	SousSousMenu1,SousSousMenu2,SousSousMenu3(>)(*),SousSousMenu4,SousSousMenu5(>) §
	SSSSMa,SSSSMa(>)(*),SSSSMc §
	OK(*),KO
}
\end{tcolorbox}

\begin{tcolorbox}[colframe=lightgray,colback=lightgray!5]
\begin{TerminalUnix}[largeur=12cm,Align=center]{}
test@DESKTOP:~$ ping -c 2 ctan.org
PING ctan.org (5.35.249.60) 56(84) bytes of data.
\end{TerminalUnix}
\end{tcolorbox}

\begin{tcolorbox}[colframe=lightgray,colback=lightgray!5]
\begin{VisionnIMG}[Largeur=12cm]{}
\includegraphics[width=5cm]{example-image-b}
\end{VisionnIMG}
\end{tcolorbox}

\begin{tcolorbox}[colframe=lightgray,colback=lightgray!5]
\begin{TerminalMac}[Largeur=14cm,Align=flush left]{}
[test@server]$ ping -c 2 ctan.org
PING ctan.org (5.35.249.60) 56(84) bytes of data.
\end{TerminalMac}
\end{tcolorbox}

\vfill~

\pagebreak

\phantomsection

\hypertarget{matoc}{}

\tableofcontents

\vspace*{5mm}

\hrule

\vspace*{5mm}

\section{Introduction}

\subsection{Description}

Avec ce packages, il est possible de créer un \og faux \fg{} terminal, ou un \og faux \fg{} menu contextuel, ou encore une \og fausse \fg{} visionneuse, comme sur un écran d'ordinateur !

Les styles sont \textit{globalement} fixés, mais quelques personnalisations sont possibles.

\subsection{Chargement}

Pour charger le package, il suffit d'ajouter dans le préambule :

\begin{quote}
\begin{verbatim}
\usepackage{sim-os-menus}
\end{verbatim}
\end{quote}

Les packages utiles, et chargés sont :

\begin{itemize}
	\item \MontreCode{tikz} (avec \MontreCode{calc,positioning}), \MontreCode{pgf}, \MontreCode{pgffor} ;
	\item \MontreCode{calc}, \MontreCode{fontawesome5}, \MontreCode{forest} (avec \MontreCode{edges}) ;
	\item \MontreCode{simplekv}, \MontreCode{xintexpr}, \MontreCode{listofitems}, \MontreCode{xstring} ;
	\item \MontreCode{settobox}, \MontreCode{tabularray} ;
	\item \MontreCode{tcolorbox} (avec \MontreCode{breakable,fitting,skins,listings,listingsutf8,hooks}).
\end{itemize}

\vfill

\subsection{Historique}

\begin{quote}
\begin{verbatim}
0.1.2 : Arborescence type explorateur
0.1.2 : Visionneuse 'like' de scripts
0.1.1 : Ajout des commandes en [fr]
0.1.0 : Version initiale
\end{verbatim}
\end{quote}

\pagebreak

\section{Les commandes disponibles}

\subsection{Menu contextuel}

Dans le but de créer un menu contextuel, la commande est :

\begin{quote}
\begin{verbatim}
%----menu contextuel
\MenuContextuel[clés]{liste des items}<options tikz>
\end{verbatim}
\end{quote}

Les \textsf{clés}, optionnelles et entre \MontreCode{[...]}, sont :

\begin{itemize}
	\item \MontreCode{CoulFond} := couleur du fond ;
	\item \MontreCode{CoulSurlig} : = couleur du surlignage ;
	\item \MontreCode{Arrondi} := booléen pour arrondir les coins (\MontreCode{true} par défaut) ;
	\item \MontreCode{Police} := police pour les items (\MontreCode{\textbackslash normalsize\textbackslash normalfont} par défaut) ;
	\item \MontreCode{CoulItems} := couleur(s) des items (\MontreCode{black} par défaut) ;
	\item \MontreCode{MargeV} := marge verticale des lignes (\MontreCode{6pt} par défaut) ;
	\item \MontreCode{MargeH} := marge horizontale des lignes (\MontreCode{12pt} par défaut) ;
	\item \MontreCode{Fleche} := caractère pour la flèche (\MontreCode{\textbackslash faCaretRight} par défaut) ;
	\item \MontreCode{ListeSeps} := liste des éventuels séparateurs (vide ou pour tous les niveaux !) ;
	\item \MontreCode{ListeIcones} := liste des éventuelles icones  (vide ou pour tous les niveaux/items !) ;
	\item \MontreCode{ListeDecals} := liste des éventuels décalages verticaux pour les sous-menus (à partir du n°2, \ldots !) (vide ou pour tous les sous-niveaux !) ;
	\item \MontreCode{Icones} := booléen pour afficher les icones (\MontreCode{false} par défaut) ;
	\item \MontreCode{Barre} := booléen pour une petite barre verticale après les éventuelles icones small vertical bar with icons (\MontreCode{true} par défaut) ;
	\item \MontreCode{Space} := espacement horizontal entre les sous-menus (\MontreCode{-0.125} par défaut).
\end{itemize}

\medskip

L'argument obligatoire, entre \MontreCode{\{...\}}, correspond à la liste des items, sous la forme :

\smallskip

\hfill\MontreCode{item1A,item1B,... § item2A,itemp2B,... § ...}\hfill~

\begin{itemize}
	\item si un item se termine par \MontreCode{(*)}, cela correspond au \textit{début} du sous-menu suivant (\underline{un seul} par niveau !) ;
	\item si un item se termine par \MontreCode{(>)} (\underline{avant} un éventuel \MontreCode{(*)}), une \textit{flèche} est écrite à la fin de la ligne de l'item.
\end{itemize}

Une syntaxe correcte est fondamentale pour que le code présente correctement le menu contextuel !

\vspace{5mm}

Quelques conseils, du fait des \textsf{clés sensibles} \MontreCode{ListeIcones}, \MontreCode{ListeDecals} et \MontreCode{ListeSeps} :

\begin{itemize}
	\item \MontreCode{ListeIcones} doit avoir le même nombre d'éléments que le nombre de menus/items (avec des items vides possibles) ;
	\item \MontreCode{ListeSeps} doit avoir le même nombre d'éléments que le nombre de niveaux (avec des niveaux vides possibles) ;
	\item \MontreCode{ListeDecals} doit avoir le même nombre d'éléments que le nombre de sous-menus (Avec 0 si pas de décalage !).
\end{itemize}

\begin{tcblisting}{colframe=lightgray,colback=lightgray!5}
%style par défaut
\MenuContextuel{Item1A(>),Item1B,Item1C(>)(*),Item1D § Item2A(*),Item2B(>),Item2C}
\end{tcblisting}

\begin{tcblisting}{colframe=lightgray,colback=lightgray!5}
%style personnalisé
\MenuContextuel[Arrondi=false,CoulFond=yellow,CoulSurlign=teal,%
  ListeSeps={1,3/2},CoulItems={black/white},Icones,Espacement=0,%
  ListeIcones={\faCut,\faIcon[regular]{copy},\faCopy,\faPaste / \faNotEqual}]
  {Item1A(>)(*),Item1B,Item1C(>),Item1D § Item2A(*),Item2B(>),Item2C}
\end{tcblisting}

\begin{tcblisting}{colframe=lightgray,colback=lightgray!5}
\MenuContextuel[Police=\sffamily,ListeSeps={1,3/1/4/1/},ListeDecals={1,2,1,0}]{%
  Menu1(>),Menu2,Menu3(>)(*),Menu4 §
  SousMenuA(>),SousMenuB(>)(*) §
  SousSousMenu1,SousSousMenu2,SousSousMenu3(>)(*),SousSousMenu4,SousSousMenu5(>) §
  SSSSMa,SSSSMa(>)(*),SSSSMc §
  OK(*),KO
}
\end{tcblisting}

\pagebreak

\subsection{Terminal}

Pour créer un \og faux \fg{} terminal (Win/UNiX/Mac), les environnements sont :

\begin{quote}
\begin{verbatim}
%----Terminal Windows
\begin{TerminalWin}[clés]{options tcbox}
    ...
\end{TerminalWin}

%----Terminal UNiX
\begin{TerminalUnix}[clés]{options tcbox}
    ...
\end{TerminalUnix}

%----Terminal OSX
\begin{TerminalMac}[clés]{options tcbox}
    ...
\end{TerminalMac}
\end{verbatim}
\end{quote}

Les \textsf{clés}, optionnelles et entre \MontreCode{[...]}, sont :

\begin{itemize}
	\item \MontreCode{Titre} := titre du terminal (\MontreCode{Terminal Win/UNiX/Mac} par défaut) ;
	\item \MontreCode{Align} := alignement horizontal (paragraphe) du terminal (\MontreCode{center} par défaut) ;
	\item \MontreCode{Largeur} := largeur du terminal (\MontreCode{\textbackslash linewidth} par défaut) ;
	\item \MontreCode{Icones} := booléen pour afficher des icones (\MontreCode{true} par défaut).
\end{itemize}

\medskip

L'argument obligatoire, entre \MontreCode{\{...\}}, correspond à des options spécifiques, à passer à la \textsf{tcolorbox}.

\begin{tcblisting}{colframe=lightgray,colback=lightgray!5}
\begin{TerminalWin}{}
Microsoft Windows [version 10.0.22000.493]
(c) Microsoft Corporation. Tous droits réservés.
C:\Users\test>ping ctan.org
Envoi d'une requête 'ping' sur ctan.org [5.35.249.60] avec 32 octets de données :
Réponse de 5.35.249.60 : octets=32 temps=35 ms TTL=51
Réponse de 5.35.249.60 : octets=32 temps=37 ms TTL=51
Réponse de 5.35.249.60 : octets=32 temps=35 ms TTL=51
Réponse de 5.35.249.60 : octets=32 temps=39 ms TTL=51
Statistiques Ping pour 5.35.249.60:
Paquets : envoyés = 4, reçus = 4, perdus = 0 (perte 0%),
Durée approximative des boucles en millisecondes :
Minimum = 35ms, Maximum = 39ms, Moyenne = 36ms
\end{TerminalWin}
\end{tcblisting}

\begin{tcblisting}{colframe=lightgray,colback=lightgray!5}
\begin{TerminalUnix}[Align=flush right]{hbox}
test@DESKTOP:~$ ping -c 2 ctan.org
PING ctan.org (5.35.249.60) 56(84) bytes of data.
\end{TerminalUnix}
\end{tcblisting}

\begin{tcblisting}{colframe=lightgray,colback=lightgray!5}
\begin{TerminalMac}[Largeur=14cm,Align=flush left]{}
[test@server]$ ping -c 2 ctan.org
PING ctan.org (5.35.249.60) 56(84) bytes of data.
\end{TerminalMac}
\end{tcblisting}

\pagebreak

\subsection{Visionneuses}

Pour créer une \og fausse\fg{} visionneuse, les environnements sont :

\begin{quote}
\begin{verbatim}
%----Visionneuse PDF
\begin{VisionnPDF}[clés]{options tcbox}
    ....
\end{VisionnPDF}

%----Visionneuse d'images
\begin{VisionnIMG}[clés]{options tcbox}
    ....
\end{VisionnIMG}

%----Visionneuse de script python
\begin{VisionnPY}[clés]{options tcbox}
    ....
\end{VisionnPY}
\end{verbatim}
\end{quote}

Les \textsf{clés}, optionnelles et entre \MontreCode{[...]}, sont :

\begin{itemize}
	\item \MontreCode{Titre} := titre de la visionneuse par défaut ;
	\item \MontreCode{Align} := alignement horizontal (paragraphe) de la visionneuse (\MontreCode{center} par défaut) ;
	\item \MontreCode{AlignH} := alignement horizontal (du contenu) de la visionneuse (\MontreCode{center} par défaut) ;
	\item \MontreCode{Largeur} := largeur de la visionneuse (\MontreCode{\textbackslash linewidth} par défaut) ;
	\item \MontreCode{Icones} := booléen pour afficher des icones (\MontreCode{true} par défaut).
\end{itemize}

\medskip

L'argument obligatoire, entre \MontreCode{\{...\}}, correspond à des options spécifiques, à passer à la \textsf{tcolorbox}.

\begin{tcblisting}{colframe=lightgray,colback=lightgray!5}
\begin{VisionnPDF}{hbox}
\fbox{\includegraphics[page=35,width=6cm]{ProfLycee-doc.pdf}}%
\fbox{\includegraphics[page=36,width=6cm]{ProfLycee-doc.pdf}}
\end{VisionnPDF}
\end{tcblisting}

\begin{tcblisting}{colframe=lightgray,colback=lightgray!5}
\begin{VisionnPDF}[Largeur=14cm,Icones=false]{}
\fbox{\includegraphics[page=65,width=4.75cm]{ProfLycee-doc.pdf}}
\end{VisionnPDF}
\end{tcblisting}

\begin{tcblisting}{colframe=lightgray,colback=lightgray!5}
\begin{VisionnIMG}[Icones=false]{hbox}
\includegraphics[width=5cm]{example-image-b}
\end{VisionnIMG}
\end{tcblisting}

\begin{tcblisting}{colframe=lightgray,colback=lightgray!5}
\begin{VisionnIMG}[Largeur=12cm]{}
\includegraphics[width=3cm]{example-image}
\end{VisionnIMG}
\end{tcblisting}

\begin{tcblisting}{colframe=lightgray,colback=lightgray!5}
%avec un environnement de présentation de codes si nécessaire
\begin{VisionnPY}[Largeur=12cm]{}
\begin{lstlisting}%
    [
        language=python,basicstyle=\ttfamily\small,
        keywordstyle=\color{green!50!black},tabsize=4,
        keywordstyle={[2]\color{magenta}},
        numbers=left,numbersep=3mm,xleftmargin=5mm,
        aboveskip=0pt,belowskip=0pt,
        numberstyle=\footnotesize\ttfamily\color{gray}
    ]
nterms = int(input("Entrez un nombre: "))

n1 = 0
n2 = 1

print("\n la suite Fibonacci est : ")
print(n1, ",", n2, end=", ")

for i in range(2, nterms):
    suivant = n1 + n2
    print(suivant, end=", ")

n1 = n2
n2 = suivant
\end{lstlisting}
\end{VisionnPY}
\end{tcblisting}

\pagebreak

\subsection{Arborescence de dossiers/fichiers}

Le package \MontreCode{forest}, et sa librairie \MontreCode{edges}, permet de présenter des arbres, à la manière d'une arborescence de fichiers.

\begin{quote}
\begin{verbatim}
\begin{ArborDossiers}[clés]{options forest}
    [dossier,ADdos
        [sous-dossier,ADdos]
        [sous-dossier,ADdos
            [fichier1,ADfic]
            [fichier2,ADfic]
        ...
    ]
        ...
    ]
\end{ArborDossiers}
\end{verbatim}
\end{quote}

Les \textsf{clés}, optionnelles et entre \MontreCode{[...]}, sont :

\begin{itemize}
	\item \MontreCode{Police} := police des labels ;
	\item \MontreCode{CoulIcoD} := couleur des icones \textsf{dossier} (\MontreCode{gray} par défaut) ;
	\item \MontreCode{CoulIcoF} := couleur des icones \textsf{fichier} (\MontreCode{gray} par défaut) ;
	\item \MontreCode{AffIcoD} := booléen pour afficher les icones \textsf{dossier} ;
	\item \MontreCode{AffIcoF} := booléen pour afficher les icones \textsf{fichier} ;
	\item \MontreCode{EspaceV} := espace vertical entre les items (\MontreCode{0.15em} par défaut) ;
	\item \MontreCode{IcoD} := icone par défaut \textsf{dossier} ;
	\item \MontreCode{IcoF} := icone par défaut \textsf{fichier}.
\end{itemize}

\medskip

Les \textsf{dossiers} sont à déclarer sous la forme \MontreCode{<nom\_dossier>,ADdos}.

Les \textsf{fichiers} sont à déclarer sous la forme \MontreCode{<nom\_fichier>,ADfic}.

\medskip

L'argument obligatoire, entre \MontreCode{\{...\}}, correspond à des options spécifiques, à passer en langage \textsf{forest}.

\begin{tcblisting}{colframe=lightgray,colback=lightgray!5}
\begin{ArborDossiers}{}
    [dossier,ADdos
        [sous-dossier,ADdos]
        [sous-dossier,ADdos
            [fichier1,ADfic]
            [fichier2,ADfic]
        ]
        [fichier3,ADfic]
    ]
\end{ArborDossiers}
\end{tcblisting}

\begin{tcblisting}{colframe=lightgray,colback=lightgray!5}
\begin{ArborDossiers}%
    [Police=\sffamily,CoulIcoD=yellow!50!pink,AffIcoF,CoulIcoF=teal,EspaceV=0.5em]%
    {}
    [dossier,ADdos
        [sous-dossier,ADdos]
        [sous-dossier,ADdos
            [fichier1,ADfic]
            [fichier2,ADfic]
        ]
        [fichier3,ADfic]
    ]
\end{ArborDossiers}
\end{tcblisting}

Les styles sont définis de manière globale, et ils peuvent être redéfinis si besoin.

\begin{tcblisting}{colframe=lightgray,colback=lightgray!5}
\tikzset{ADtrait/.style={thick,->,densely dashed,red,>=latex}}
\begin{ArborDossiers}%
    [Police=\sffamily,CoulIcoD=yellow!50!pink,AffIcoF,CoulIcoF=teal,EspaceV=0.5em]%
    {l sep=2em}
    [dossier,ADdos
        [sous-dossier,ADdos]
        [sous-dossier,ADdos
            [fichier1,ADfic]
            [fichier2,ADfic]
        ]
        [fichier3,ADfic]
    ]
\end{ArborDossiers}
\end{tcblisting}

\end{document}